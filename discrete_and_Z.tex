\documentclass{report}
\usepackage[T2A]{fontenc}
\usepackage[utf8]{inputenc}
\usepackage[russian]{babel}
\title{Разностные уравнения и D-преобразование}
\author{Алексей Макашов}
\date{2019}
\begin{document}
    \maketitle
    \tableofcontents

    \chapter{Дискретное преобразование Лапласа и $Z$-преобразование}

    \chapter{Преобразование $\bar{D}$ и $\bar{Z}$}

    Для установления связи между непрерывным и дискретным преобразованием 
    Лапласа
    воспользуемся определением дискретной функции. Итак, пусть $x_T(t)$ - 
    непрерывная функция. Выполним замену переменной и обозначим $t=\tau T$ и
    обозначим $x(\tau)=x_T(T\tau)$. Пусть, кроме того, для непрерывной функции
    $x(\tau)$ существует изображение по Лапласу:

    \begin{equation}[laplace_direct]
        X(p)=L\{x(\tau)\} = \int_0^{\infty} e^{-p\tau}x(\tau) d\tau
    \end{equation}
     
Перейдём к дискретной функции смещённого аргумента. Для этого, как было показано в ???,
перейдём к переменным $n, \varepsilon$ так, что

\begin{equation}[discrFunc]
    x[n,\varepsilon]= x(\tau)|_{\tau=n+\varepsilon}, \quad 0\leq\varepsilon\le  1
\end{equation}
и запишем дискретное преобразование Лапласа:
\begin{equation}\label{lapDiscrete}
    X^*(q,\varepsilon)=\sum_{n=0}^{\infty} e^{-nq}x[n,\varepsilon]
\end{equation}

\end{document}
